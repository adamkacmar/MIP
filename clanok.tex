% Metódy inžinierskej práce

\documentclass[10pt,twoside,slovak,a4paper]{article}

\usepackage[slovak]{babel}
%\usepackage[T1]{fontenc}
\usepackage[IL2]{fontenc} % lepšia sadzba písmena Ľ než v T1
\usepackage[utf8]{inputenc}
\usepackage{graphicx}
\usepackage{url} % príkaz \url na formátovanie URL
\usepackage{hyperref} % odkazy v texte budú aktívne (pri niektorých triedach dokumentov spôsobuje posun textu)

\usepackage{cite}
%\usepackage{times}

\pagestyle{headings}

\title{Vplyv násilnych hier na správanie človeka\thanks{Semestrálny projekt v predmete Metódy inžinierskej práce, ak. rok 2022/23, vedenie: Ing. Vladimír Mlynarovič, PhD.}} % meno a priezvisko vyučujúceho na cvičeniach

\author{Adam Kačmár\\[2pt]
	{\small Slovenská technická univerzita v Bratislave}\\
	{\small Fakulta informatiky a informačných technológií}\\
	{\small \texttt{xkacmara@stuba.sk}}
	}

\date{\small 24. október 2022} % upravte



\begin{document}

\maketitle

\begin{abstract}
Článok sa zaoberá násilím vo videohrách a jeho vplyvom na hráčov počas hrania v online svete ako aj v reálnom živote. Bežný stereotyp rodičov detí v mladom veku, ktorý počul každý hráč pri hraní strieľačiek nám predostiera otázku či tomu tak naozaj je. Herný svet v dnešnej dobe ponúka množstvo variácií hier, ktoré nás dostávajú do pozícií vojakov, nájomných vrahov či obrancov vesmíru. Takéto široké spektrum nám ponúka aj možnosť herného zážitku pre viacerých hráčov a kompetitívny mód, pri ktorom sa dokáže hráč rozčúliť nad slabším výkonom spoluhráča či tímu a dá im to ľudovo „zožrať“. Pri veľkej dávke súťaživosti a vžitia sa do hry rastie rovnako aj agresia hráča. Otázkou tohto článku je či takáto agresia je vyvolaná práve typom hry, ktorá násilie podsúva. Výsledky nám dávajú štúdie, ktoré merali správanie účastníkov pri viacerých scenároch.
\end{abstract}



\section{Úvod}

V dnešnej dobe býva bežnou praxou, že sú videohry súčasťou voľného času mladých ľudí, pri ktorých sa odpútajú od reality počas hier s priateľmi. Počas dekád vývoja herného priemyslu sa vyvinuli žánre a herná realita do nepredstaviteľných rozmerov, u ktorých si ľudia v minulosti nedokázali predstaviť realizáciu. V posledných 15 rokoch takého technológie stúpli na nový výkonnostný level reality a s tým sa aj hry variabilných žánrov vykryštalizovali takmer k dokonalosti.

Jednými zo žánrov sú aj akčné a first-person shooter hry, ktoré sú založené najmä na násilí voči nepriateľom v rôznych formách. Priamoúmerne s vývojom a modernizáciou videohier sa vizuálne približujú aj „strieľačký“ bližšie k reálnemu svetu. Pri týchto hrách je bežná herná možnosť módu pre viacerých hráčov, resp. multiplayer, v ktorom sa súťaživá nenávisť voči protihráčom premieňa na nenávisť z frustrácie voči spoluhráčov, ktorých výkon nenapĺňa ideálne predstavy hráča. Podobným nenávistným a násilným prejavom dokáže čeliť aj rodina a okolie hráča, v zriedkavých prípadoch aj udalosti, ktoré pohnú spoločnosťou, štátom aj svetom. 

V minulosti sa spoločnosť stretla práve s touto otázkou pri masových vraždách, streľbách alebo teroristických činoch, za ktorými stáli mladí ľudia, u ktorých sa domnievalo, že dôležitým faktorom boli aj násilné videohry, ktoré pridávali myšlienky konateľom smrteľných aktov ~\ref{ina:minulost}. Práve tieto korelácie medzi násilnými hrami, ktoré sledujú výskumy v posledných rokoch dávajú lepší drobnohľad na túto problematiku(~\ref{GAM}). Záverečne poznámky podávajú pohľad individuála a laika ako aj samostatný výsledok výskumu.

%Motivujte čitateľa a vysvetlite, o čom píšete. Úvod sa väčšinou nedelí na časti.

%Uveďte explicitne štruktúru článku. Tu je nejaký príklad.
%Základný problém, ktorý bol naznačený v úvode, je podrobnejšie vysvetlený v časti~\ref{nejaka}.
%Dôležité súvislosti sú uvedené v častiach~\ref{dolezita} a~\ref{dolezitejsia}.
%Záverečné poznámky prináša časť~\ref{zaver}.


\section{Prejavy násilia a nenávisti} \label{prejavy}

Prejavy násilia a nenávisti sú bežnou praxou počas hrania videohier. Takéto prejavy sa diferencujú aj podľa typu danej hry na offline hry a online hry. Každa subkategória má u svojich hier rôzne prejavy násilia a nenávisti, ktoré vznikajú frustráciou z priebehu hry, výkonov hráča alebo spoluhráčov alebo iných faktorov. Frustrácia pôsobí aj pri neschopnosti hráča ovládať hru správne alebo neschopnosti zlepšovať sa v nej\cite{UoR-Failure}, prevažne pri offline hrách. Pri online hrách sa pretláča aj nenávisť voči spoluhráčom, na ktorú poukazuje pasáž ~\ref{ina:spoluhraci}. 


\subsection{Prejavy voči spoluhráčom} \label{ina:spoluhraci}
Prejavy nenávisti pri móde pre viacerých hráčov sú už štandardom. Hráč počas online hrania prichádza do mnohých situácií, kde je dôležitá kooperácia so spoluhráčmi, ktoré môžu výrazne ovplyvniť priebeh hry. V prípade neúspechu a zlyhania jednotlivca sa dostávajú hráči do vzájomných sporov o tom, kto chybu spôsobil. Následná slovná výmena cez chat alebo audioprenos eskaluje až do hraničných situácií, kde téma urážok z oboch strán nemá už nič spoločné s hrou samotnou. 

Pribudajú vulgarizmy rôznych variánt, ktoré sú neetické a vedeli by aj hraničiť so zákonom. Frustrovaný hráč vyjadruje svoje vulgarizmy vo forme nenávistných poznámok na blízku rodinu dotknutej osoby, rodičov, súrodencov či dokonca ešte nenarodené deti. Ďalšou formou sú národnnostné poznámky, pri ktorých sa vytiahnú tie najohranejšie naratívy a stereotypy pre daný národ. Do väčšieho a hraničného extrému zapadajú už len rasistické vulgarizmy, sexistické stereotypy alebo prebúdzanie myšlienok nelegálnych ideológií.

S väčšou hráčskou základňou pribúda väčší počet konfliktov, ktoré vznikajú. K máju 2022 sa v rebríčku 15 najpopulárnejších hier(\cite{TopGames}) roku ocitlo až 12 hier s explicitným obsahom, ktoré sú zároveň aj multiplayerové. Pre herné spoločenstvo nie je žiadnym prekvapením, že sú medzi nimi hry ako Counter Strike: Global Offensive, League of Legends alebo Fortnite, ktoré sú verejné známe ako hry s vysokým podielom toxického správania a nenávisti v komunite.

\subsection{Prejavy voči okoliu} \label{ina:okolie}
Rovnako ako sú bežné prejavy frustrácie na spoluhráčov vedia byť bežné aj prejavy frustrácie na svojom okolí. Slovné prejavy sú však miernejšie a nezachádzajú do extrémov. Rodičia, súrodenci či spolubývajúci dostanú neprijemnú odpoveď s agresivným a nahnevaným podtónom. 

V prípade, že sa takáto agresia neprejaví voči človeku, vo väčšine prípadov si hráč vybije hnev na okolitých predmetoch. V dôsledku toho dostáva počítačové príslušenstvo, nábytok či stena do pozície svojho terču. V niektorých prípadoch hráči vybíjajú svoju zlosť aj sami na sebe: údermi do nôh, hrude alebo kopaním nôh do tvrdých predmetoch.

\subsection{Udalosti v minulosti} \label{ina:minulost}
Otázky ohľadom vplyvu násilných hier boli v minulost pokládané po tragických akciách

\begin{itemize}
\item jedna vec
\item druhá vec
	\begin{itemize}
	\item x
	\item y
	\end{itemize}
\end{itemize}

Ten istý zoznam, len číslovaný:

\begin{enumerate}
\item jedna vec
\item druhá vec
	\begin{enumerate}
	\item x
	\item y
	\end{enumerate}
\end{enumerate}

\paragraph{Veľmi dôležitá poznámka.}
Niekedy je potrebné nadpisom označiť odsek. Text pokračuje hneď za nadpisom.


\section{Normatívny vplyv okolia} \label{normat}



\section{General Aggression Model} \label{GAM}
General Aggression Model (GAM) je často používaným modelom, ktorý sa využíva ako ukazovateľ vo vzťahu k agresivite, v tomto prípade aj spojenú s videohrami, prejavovanou v reálnom svete. Zaoberá sa úlohou sociálnych, kognitívnych, osobnostných, vývinových a biologických faktorov na agresivitu. Približné procesy GAM podrobne opisujú, ako faktory osoby a situácie ovplyvňujú kognície, pocity a vzrušenie, ktoré následne ovplyvňujú procesy hodnotenia a rozhodovania, ktoré následne ovplyvňujú agresívne alebo neagresívne výsledky správania. Každý cyklus približných procesov slúži ako skúška učenia, ktorá ovplyvňuje vývoj a dostupnosť agresívnych znalostných štruktúr\cite{GAM}. 


\section{Záver} \label{zaver} % prípadne iný variant názvu



%\acknowledgement{Ak niekomu chcete poďakovať\ldots}


% týmto sa generuje zoznam literatúry z obsahu súboru literatura.bib podľa toho, na čo sa v článku odkazujete
\bibliography{literatura}
\bibliographystyle{plain} % prípadne alpha, abbrv alebo hociktorý iný
\end{document}
